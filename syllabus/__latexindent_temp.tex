\documentclass[11pt]{article}

  % Package declarations
  \usepackage{advdate}% http://ctan.org/pkg/advdate
  \usepackage{amsmath}
  \usepackage{array}
  \usepackage{color}
  \usepackage[mmddyyyy]{datetime}% http://ctan.org/pkg/datetime
  \usepackage{fancyhdr}
  \usepackage{float}
  \usepackage[T1]{fontenc}
  \usepackage{fullpage}
  \usepackage[left=1in,top=1in,right=1in,bottom=1in,headheight=3ex,headsep=3ex]{geometry}
  \usepackage{graphicx}
  \usepackage{hyperref}
  \usepackage{lastpage}
  \usepackage{layout}
  \usepackage[sc]{mathpazo}
  \usepackage{multicol}
  \usepackage{setspace}
  \usepackage{url}
  \usepackage{xcolor}

  % Layout settings
  \pagestyle{fancy}
  \linespread{1.05} % Palatino needs more leading (space between lines)
  \definecolor{nyupurple}{HTML}{57068C}
  \hypersetup{colorlinks,breaklinks,linkcolor=nyupurple,urlcolor=nyupurple,anchorcolor=nyupurple,citecolor=black}

  % New commands and modifications
  \newcommand{\blankline}{\quad\pagebreak[2]}
  \newdateformat{syldate}{\twodigit{\THEMONTH}/\twodigit{\THEDAY}}
  \newsavebox{\MONDAY}\savebox{\MONDAY}{Mon}% Mon
  \newcommand{\week}[1]{%
    \paragraph*{\kern-2ex\quad #1, \syldate{\today} - \AdvanceDate[4]\syldate{\today}:}% Set heading  \quad #1
    \ifdim\wd1=\wd\MONDAY
      \AdvanceDate[7]
    \else
      \AdvanceDate[7]
    \fi%
  }

  \newcommand{\module}[1]{%
    \paragraph*{\kern-2ex\quad #1:}
  }


  % MODIFY: Document header and footer
  \lhead{}
  \chead{}
  \rhead{\footnotesize Computational Social Science}
  \lfoot{}
  \cfoot{\small \thepage/\pageref*{LastPage}}
  \rfoot{}

  % MODIFY: The course title and professor
  \title{Data Skills for Computational Social Science}
  \author{Chase Coleman, Spencer Lyon, and Thomas Sargent}
  \date{Fall 2021}


\begin{document}

\maketitle

\blankline

\begin{tabular*}{.93\textwidth}{@{\extracolsep{\fill}}lr}

  % Modify information %%%%%%%%%%%%%%%%%%%%%%%%%%%%%%%%%%%%%%%%%
  E-mail: \texttt{cc7768@gmail.com} & Web: \href{http://www.chasegcoleman.com}{\tt\bf http://www.chasegcoleman.com}  \\
  E-mail: \texttt{spencerlyon2@gmail.com} & Web: \href{http://www.spencerlyon.com}{\tt\bf http://www.spencerlyon.com}  \\
  E-mail: \texttt{ts43@nyu.edu} & Web: \href{http://www.tomsargent.com/}{\tt\bf http://www.tomsargent.com/}  \\

  Office Hours: By appointment  &  Class Hours: Thursday 18:00-20:50 \\

  \hline

\end{tabular*}

\vspace{5 mm}


%
% Course description
%
\section*{Course Description}

  This course teaches the foundational skills necessary to do modern data
  analytics using the Python programming language. We assume that students have
  previously worked with Python. We will add to existing Python skills and teach
  the core scientific and data-specific libraries (numpy, scipy, matplotlib, and
  pandas). We will use these skills to analyze a variety of social science
  datasets and answer research and business questions.

%
% Course materials
%
\section*{Course Materials}

  \begin{itemize}
    \item {\bf QuantEcon Datascience} Lectures from the QuantEcon datascience
    sequence at \newline\href{https://datascience.quantecon.org}{\tt\bf
    https://datascience.quantecon.org}
    \item {\bf Python Data Science Handbook} by Jake Vanderplas\newline\href{https://jakevdp.github.io/PythonDataScienceHandbook/}{\tt\bf https://jakevdp.github.io/PythonDataScienceHandbook/}
    \item {\bf Python for Data Analysis, 2nd Edition} by Wes McKinney
  \end{itemize}


%
% Prerequisites
%
\section*{Prerequisites}

  Students should have prior experience using the Python programming language
  before starting this course. Ideally, students will have completed the Summer
  "Pre-Course" that teaches these skills. Throughout this course we will
  leverage programming skills such as control flow constructs (\texttt{if/else},
  \texttt{for/while}), defining custom functions (\texttt{def}), and finding
  help on existing functions (\texttt{?} in Jupyter environments and
  \texttt{help} elsewhere).

  \bigskip

  \noindent Althoughy a course in probability or statistics is not a prerequisite, students will
  find some knowledge of these topics to be helpful.

  \bigskip

  \noindent Students are required to have access to a personal computer (laptop)
  that can be brought with them to lectures. Computers will be used by students
  in each class session and those without a laptop will not get the necessary
  in-class practice in order to master the concepts we study.

%
% Course objectives
%
\section*{Course Objectives}

  The key objective in this course is for students to use Python to do
  meaningful data analysis using social science data sets. Success in this
  course can be described as the student's ability to do the following:

  \begin{itemize}
    \item Read, write, and understand basic programs written in the Python
    programming language
    \item Import, clean, combine, and summarize datasets from a variety of sources
    \item Construct informative visualizations of raw data and model results
    \item Implement basic data engineering best practices such as optimizing
    organization and structure of datasets, using effective storage formats for
    a given task, and automating repetitive extract-transform-load (ETL)
    processes
  \end{itemize}

%
% Course structure
%
\section*{Course Structure}

  \subsection*{Class Structure}

    This course will meet once a week for 3 hours.

    \bigskip

    \noindent Class will be treated as a mixture of lecture time and lab time.
    Students should bring, and expect to use, their laptops every time the class
    meets.


  \subsection*{Assessments}

    This course will use a mixture of homework assignments, in-class quizzes,
    exams, and a final project to evaluate students.

    \bigskip

    \noindent \textbf{Homework}:  At the beginning of the course, homework  will
    be assigned almost every week. Later in the course, there will be less
    frequent assignments in order to assure that you have time to work on your
    class project. Your two lowest homework grades will be dropped.

    \noindent \textbf{In-class participation}: As the class will be offered virtually,
    being an active participant in lectures and discussion requires effort. We recognize
    and appreciate this effort and will reward it accordingly.

    \noindent \textbf{Exams}: There will be 1 take-home exam.

    \noindent \textbf{Project}: There will be a class project aimed at helping
    you apply the tools that you have learned to a ``real-world problem.''

    \vspace{5mm}

    \noindent Other than for the exam, we highly encourage students to
    work together. We have found that groups of 3-4 seem to work best. We
    believe that collaborative work is the best way to learn the type of
    material that we cover.  We advise  students not to rely on others to do
    work that you do not understand.


  \subsection*{Grading Policy}

    The assignments just described will be the main inputs to the grade for the
    course. Assignments will be weighted evenly within groups and overall
    according to the following decision rule:

    \begin{itemize}
      \item Homework assignments: 25\%
      \item In-class participation: 15\%
      \item Tests: 20\%
      \item Project: 40\%
    \end{itemize}

    \noindent This weighting reflects our opinion that the most important skills
    to be  acquired in this class are communicated by one's ability successfully
    to apply the tools that you learn to an interesting question in the social
    sciences.

%
% Course schedule
%
\newpage
\section*{Schedule and weekly learning goals}

  The schedule is tentative and subject to change. Several of the modules below will occupy more than one week.  The learning goals target
  key concepts to be mastered after each module. Successive modules build on early modules.

  \module{Part 1} Introduction to Pandas

  \bigskip

  {\bf Sources and tools:}
  \begin{itemize}
    \item Class notes
    \item \texttt{pandas package:} \url{https://pandas.pydata.org/}
    \item \url{https://python.quantecon.org/pandas.html}
    \item \url{https://datascience.quantecon.org/pandas/intro.html}
    \item \texttt{pandas.DataFrame}
    \item \texttt{pandas.Series}
    \item Chapter 5 of Python for Data Analysis
  \end{itemize}

  {\bf Topics to be mastered:}
  \begin{itemize}
    \item Pandas datatypes: DataFrame and Series
    \item Basic operations with DataFrames: summary statistics, aggregations,
    transformations, data selection
    \item Sorting and ranking
    \item Value counts
    \item Function application and mapping
    \item Duplicate labels
    \item Basic visualization using the \texttt{plot} method
  \end{itemize}

  \module{Part 2} Organizing Data With Pandas, I

  \bigskip

  {\bf Sources and tools:}
  \begin{itemize}
    \item Class notes
    \item \texttt{pandas.Index}
    \item \url{https://datascience.quantecon.org/pandas/the_index.html}
    \item \url{https://datascience.quantecon.org/pandas/storage_formats.html}
    \item Chapter 6 of Python for Data Analysis
  \end{itemize}

  {\bf Topics to be mastered:}
  \begin{itemize}
    \item Understanding the \texttt{Index} in pandas
    \item Storage formats
    \item Reindexing
    \item Stacking and melting
    \item Hierarchical indexing
  \end{itemize}

  \module{Part 3} Organizing Data With Pandas, II

  \bigskip

  {\bf Sources and tools:}
  \begin{itemize}
    \item Class notes
    \item \url{https://datascience.quantecon.org/pandas/data_clean.html}
    \item \url{https://datascience.quantecon.org/pandas/reshape.html}
    \item Chapter 7 and 8 of Python for Data Analysis
  \end{itemize}

  {\bf Topics to be mastered:}
  \begin{itemize}
    \item Cleaning, reshaping, and merging datasets
    \item \texttt{merge}, \texttt{join} and \texttt{combine}
    \item Stacking and melting
    \item Handling missing data
    \item Discretization and binning
    \item Random sampling
    \item String manipulation
  \end{itemize}

  \module{Part 4} Grouped Operations with Pandas, I

  \bigskip

  {\bf Sources and tools:}
  \begin{itemize}
    \item Class notes
    \item \texttt{pandas.groupby}
    \item \url{https://pandas.pydata.org/pandas-docs/stable/user_guide/groupby.html#}
    \item \url{https://datascience.quantecon.org/pandas/groupby.html}
    \item Chapter 10 of Python for Data Analysis
  \end{itemize}

  {\bf Topics to be mastered:}
  \begin{itemize}
    \item \texttt{groupby} method with built-in methods
    \item Groupby mechanics
    \item Custom grouped functions
  \end{itemize}

  \module{Part 5} Grouped Operations with Pandas, II

  \bigskip

  {\bf Sources and tools:}
  \begin{itemize}
    \item Class notes
    \item \texttt{pandas.groupby}
    \item \url{https://pandas.pydata.org/pandas-docs/stable/user_guide/groupby.html#}
    \item \url{https://datascience.quantecon.org/pandas/groupby.html}
    \item Chapter 10  of Python for Data Analysis
  \end{itemize}

  {\bf Topics to be mastered:}
  \begin{itemize}
    \item Aggregation with multiple function application
    \item General split-apply-combine
    \item \texttt{transform}
    \item \texttt{apply}
  \end{itemize}

  \module{Part 6} Grouped Operations with Pandas, III

  \bigskip

  {\bf Sources and tools:}
  \begin{itemize}
    \item Class notes
    \item \url{https://pandas.pydata.org/pandas-docs/stable/user_guide/groupby.html#}
    \item \url{https://datascience.quantecon.org/pandas/groupby.html}
    \item Chapter 12 of Python for Data Analysis
  \end{itemize}

  {\bf Topics to be mastered:}
  \begin{itemize}
    \item Pivot tables and cross-tabulation
    \item Method chaining with \texttt{pipe}
    \item Categorical data
  \end{itemize}

  \module{Part 7} Time Series with Pandas, I

  \bigskip

  {\bf Sources and tools:}
  \begin{itemize}
    \item Class notes
    \item \url{https://pandas.pydata.org/pandas-docs/stable/user_guide/timeseries.html}
    \item \url{https://datascience.quantecon.org/pandas/timeseries.html}
    \item Chapter 11 of Python for Data Analysis
  \end{itemize}

  {\bf Topics to be mastered:}
  \begin{itemize}
    \item Rolling-window operations
    \item Resampling frequency of observations
    \item Doing arithmetic with dates, date ranges, periods, and TimeDeltas
  \end{itemize}

  \module{Part 8} Time Series with Pandas, II

  \bigskip

  {\bf Sources and tools:}
  \begin{itemize}
    \item Class notes
    \item \url{https://pandas.pydata.org/pandas-docs/stable/user_guide/timeseries.html}
    \item \url{https://datascience.quantecon.org/pandas/timeseries.html}
    \item Chapter 11 of Python for Data Analysis
  \end{itemize}

  {\bf Topics to be mastered:}
  \begin{itemize}
    \item Upsampling and interpolation
    \item Downsampling
    \item Handling time zones
  \end{itemize}

  \module{Part 9} Data Visualization, I

  \bigskip

  {\bf Sources and tools:}
  \begin{itemize}
    \item Class notes
    \item \url{https://datascience.quantecon.org/pandas/matplotlib.html}
    \item \url{https://seaborn.pydata.org/}
    \item \url{https://plot.ly/python/}
    \item \url{https://altair-viz.github.io/}
    \item Chapter 9 of Python for Data Analysis
    \item \url{https://datascience.quantecon.org/applications/maps.html}
  \end{itemize}

  {\bf Topics to be mastered:}
  \begin{itemize}
    \item Intermediate \texttt{matplotlib}
    \item Statistical visualization with \texttt{seaborn}
    \item Widgets
  \end{itemize}

  \module{Part 10} Data Visualization, II

  \bigskip

  {\bf Sources and tools:}
  \begin{itemize}
    \item Class notes
    \item Chapter 9 of Python for Data Analysis
    \item \url{https://datascience.quantecon.org/applications/maps.html}
  \end{itemize}

  {\bf Topics to be mastered:}
  \begin{itemize}
    \item Interactive web-based visualizations, and dashboards using \texttt{plotly} and
      \texttt{altair} --- As an example of what could be done, see
      \href{https://www.waugheconomics.com/trade-war-map.html}{Mike Waugh's webpage}
  \end{itemize}

  \module{Part 11} Data Harvesting

  \bigskip

  {\bf Sources and tools:}
  \begin{itemize}
    \item Class notes
    \item \url{https://scrapy.org/}
    \item \url{https://camelot-py.readthedocs.io/en/master/}
    \item \url{https://www.crummy.com/software/BeautifulSoup/bs4/doc/}
  \end{itemize}

  {\bf Topics to be mastered:}
  \begin{itemize}
    \item Integrating with Web APIs
    \item Scraping data from websites without an api (\texttt{scrapy})
    \item Extracting data from PDFs (\texttt{camelot})
    % TODO: Application is getting data on inequality etc...
  \end{itemize}

  \module{Part 12} Data Engineering

  \bigskip

  {\bf Sources and tools:}
  \begin{itemize}
    \item Class notes
    \item \url{https://airflow.apache.org/}
    \item \url{https://www.sqlalchemy.org/}
  \end{itemize}

  {\bf Topics to be mastered:}
  \begin{itemize}
    \item Basic introduction to databases (using SQLite through \texttt{sqlalchemy})
    \item Automation and data pipelines using Apache Airflow
    \item We will illustrate these tools by creating an automatically updating database on one of
      a few potential topics. Our choice of topic will depend on class interest.
  \end{itemize}

  \module{Part 13} Case studies, I

  \bigskip

  {\bf Sources and tools:}
  \begin{itemize}
    \item \url{http://www.tomsargent.com/research/ReadMe_Pub.pdf}
    \item \url{https://datascience.quantecon.org/applications/}
    \item Chapter 14 of Python for Data Analysis
    \item \url{https://datascience.quantecon.org/applications/recidivism.html}
  \end{itemize}

  {\bf Topics to be mastered:}
  \begin{itemize}
    \item Combine the tools learned in this class to generate automatically updated databases
      and visualizations, covering topics such as
    \begin{itemize}
      \item Inequality data
      \item U.S. bond data and term structure of interest rates; see Hall, Payne, Sargent bond
        dataset
    \end{itemize}
  \end{itemize}

  \module{Part 14} Case studies, II

  \bigskip

  {\bf Sources and tools:}
  \begin{itemize}
    \item \url{https://datascience.quantecon.org/applications/}
    \item Chapter 14 of Python for Data Analysis
  \end{itemize}

  {\bf Topics to be mastered:}
  \begin{itemize}
    \item Examples from
      \href{https://www.hup.harvard.edu/catalog.php?isbn=9780674237544}{\textit{The Great Reversal}}
      by Thomas Phillipon
  \end{itemize}

  \module{Part  15} Case studies, III

  \bigskip

  {\bf Topics to be mastered:}
  \begin{itemize}
    \item Student presentations of class projects
  \end{itemize}

% Set first date of the semester (for some reason this is a week before what comes up, but that's easy to get around)
% \SetDate[06/01/2020]
% \week{Week 01} Jupyter Notebook, {\bf LS, chap.~6}
% \begin{itemize}
% \item Bellman equations
% \item McCall search model
% \end{itemize}

% \week{Week 02} Time Series {\bf LS, chap.~2}
% \begin{itemize}
% \item Markov chains
% \item Linear State Space models
% \end{itemize}

% Add a figure %%%%%%%%%%%%%%%%%%%%%%%%%%%%%%%%%%%%%%%%%%%

%\begin{figure*}
%\includegraphics[width=1.3\textwidth,angle=90]{Concept_map_315.pdf}
%\end{figure*}


\newpage
\section*{Course Policies}

\subsection*{Professional Behavior}

\footnotesize{
  Attend class. They say ``eighty percent of success is just showing up.'' We have found that
  those who show up perform systematically better.
}

\footnotesize{
  Arrive to class on time and stay until the end of class. Chronically arriving late or leaving
  class early is unprofessional and disruptive to the rest of the class.
}

\footnotesize{
  We understand that the electronic recording of notes will be important for class and
  so computers will be allowed in class. Please refrain from using computers for anything but
  activities related to the class. Phones are prohibited as they are rarely useful for anything in
  the course. Eating and drinking are allowed in class but please refrain from it affecting the
  course. Try not to eat your lunch in class as the classes are typically active.
}

\end{document}
